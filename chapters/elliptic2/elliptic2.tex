\chapter{The finite element method for elliptic problems}

\section{The weak formulation and the finite element formulation}

We are now in position to analyze the finite element method
for an elliptic problem in detail and provide rigorous estimates
on the numerical accuracy. We remember the strong formuation:  
Find the solution $u$ of the problem

\begin{eqnarray}
\label{chp3:elliptic}
-\nabla\cdot(k\nabla u)  &=& f \quad \textrm{in}\ \Omega,\\
\label{chp3:Dirichlet}
u&=& g \quad \textrm{on}\ \partial\Omega_D, \\
\label{chp3:Neumann}
k \frac{\partial u}{\partial n}&=& h \quad \textrm{on}\ \partial\Omega_N . 
\end{eqnarray}
and assume $\partial \Omega = \partial \Omega_D \cup \partial \Omega_N$ 
and $\partial \Omega_D \cap \partial \Omega_N = \emptyset$.

In order to arrive at the weak formulation, we introduce two Sobolev spaces
\begin{align}
H^1_{0, D} (\Omega) &= \{ u \in H^1(\Omega)  | \  T u = 0 \mbox{ on } \partial \Omega_D \}, \\   
H^1_{g, D} (\Omega) &= \{ u \in H^1(\Omega)  | \  T u = g \mbox{ on } \partial \Omega_D \}  .  
\end{align}
Here, both spaces are subspaced of the $H^1$ space introduced in the previous chapter which 
include only the functions with appropriate values on the Dirichlet part of the boundary $\partial \Omega_D$.  
As detailed in Chapter \ref{elliptic}, we obtain the weak formulation by 1) multiplying
the equation \eqref{chp3:elliptic} with a test function $v$, 2) employing Gauss-Green's lemma, and 3) employing the 
boundary conditions. We arrive at (where we now include the precise definitions of the wereabouts of the
trial and test functions): \\
Given $f\in H^{-1}_D$, $h\in H^{-1/2}(\partial \Omega_N)$,  
find $u\in  H^1_{g, D} (\Omega)$ such that  
\begin{equation}
\label{cp3:weak} 
\int_\Omega (k \nabla u) \cdot \nabla v \, dx = \int_\Omega f v \, dx + \int_{\partial \Omega_N} h v \, ds, \quad    \forall v\in  H^1_{0, D} (\Omega).  
\end{equation} 
Here, the negative norms are similar to the negative norms introduced in the previous chapter, although we there had Dirichlet conditions
on the complete boundary. Alernatively, and perhaps simpler from a mathematical point of view is to define it in terms of duality as
in Remark \ref{dualnorm}. In implementation, we seldom need to pay attention to what space our input functions reside in, but we remark
that it is sometimes useful to include functions that are not defined in points but only through integration over elements.  

In order to define a finite element method, let $\Omega_h$ be a mesh consisting of a set $E_h$ of cells. 
For simplicity, we assume that the cells are simplices (triangles in 2D, tetrahedrons in 3D). Simplices 
fit very well with standard polynomials, so let ${\cal P}_k$ be the space of 
polynomials of order $k$, i.e., a basis in 2D could be $\{x^i y^{j} \}_{i,j | i+j \le k}$.   
A corresponding finite element method consist of some trial and test spaces 
\begin{align}
V_{h, 0}  &= \{ u \in H^1_{0,D}(\Omega) \ | \  \forall e \in E_h, u|_e \in {\cal P}_k \}, \\   
V_{h, g}  &= \{ u \in H^1_{g,D}(\Omega) \ | \  \forall e \in E_h, u|_e \in {\cal P}_k \} .  
\end{align}
\kent{perhaps instead introduce Lagrange elements} 
The corresponding finite element formulation is then 
Given $f\in H^{-1}_D$, $h\in H^{-1/2}(\partial \Omega_N)$,  
find $u_h\in  V_{h, g} $ such that  
\begin{equation}
\label{cp3:weak} 
\int_\Omega (k \nabla u_h ) \cdot \nabla v_h \, dx = \int_\Omega f \, v_h \, dx + \int_{\partial \Omega_N} h \, v_h \, ds, \quad    \forall v\in  V_{h, 0} .  
\end{equation} 

In the previous chapter we listed some approximation properties of polynomials in general. We will now employ these results in order to
derive \emph{a priori} error estimates. However, in order to achieve this we need four concepts: Poincare's inequality, lifting, the positivity of $k$
and Galerkin orthogonality. 

Poincare's inequality states that for $u\in  H^1_{0, D}(\Omega)$ we have
\begin{equation} 
\label{poincare}
\|u\|_{L^2(\Omega)}  \le C \| \nabla u\|_{L^2(\Omega)} 
\end{equation} 
We will not prove this inequality but note that it can be easily motivated by a Taylor approximation 
\[
f(x+h) \le f(x) + h \max_{x \le x^* \le x+h } \,  f'(x^*) 
\]
Chosing $x$ on $\partial \Omega_D$ makes $f(x)=0$ and hence $(f(x+h)$ can be bounded by the derivative
of $f$. Of course, Poincare's result provides a bounds on the integral rather than the pointwise sense 
of Taylor, but the results are similar. 

Lifting, which is often used for theoretical purposes but seldom in implementation, 
refers to changing the problem from having non-homogenous Dirichlet conditions to homogenous Dirichlet conditions.  
Let us assume that $G$ is any function in $H^1(\Omega)$ such that $T u |_{\partial \Omega_D} = g$. Then, by linearity, 
$u= u_0 + G$ solves \eqref{chp3:elliptic}--\eqref{chp3:Neumann} with $u_0$ defined as     
\begin{eqnarray}
\label{chp3:elliptic:lift}
-\nabla\cdot(k\nabla u_0 )  &= f -\nabla\cdot(k\nabla G)  &\quad \textrm{in}\ \Omega,\\
\label{chp3:Dirichlet:lift}
u&= 0 &\quad \textrm{on}\ \partial\Omega_D, \\
\label{chp3:Neumann:lift}
k \frac{\partial u_0}{\partial n}&= h - k \frac{\partial G}{\partial n} &\quad \textrm{on}\ \partial\Omega_N . 
\end{eqnarray}

In order for \eqref{chp3:element} to be an elliptic problem $k \in \mathbb{R}^{d\times d}$ given $\Omega\in \mathbb{R}^d$  
$k$ needs to be stricktly positive for all $x\in \Omega$. That is  
For $\forall \xi\in \mathbb{R}^n$ $\forall x \in \Omega$ there exists a $k_0 > 0$ such that 
\[
\xi^T k \, \xi > k_0 |\xi|^2 .   
\]

Given that Poincare's inequality applies and that $k$ is stricktly positive, as defined above, 
\[
\int_\Omega (k \nabla u) \cdot \nabla v \, dx 
\]
defines an inner product 


\begin{exercise}
\label{ex:poincare}

Let $\Omega=(0,1)$ then  
for all functions in $H^1_0(\Omega)$
Poincar\'e's inequality states that
\[
|u|_{L^2} \le C  |\frac{\partial u}{\partial x}|_{L^2}   
\]
Use this inequality to show that the $H^1$ semi-norm defines 
a norm equivalent with the standard $H^1$ norm on $H^1_0(\Omega)$.  
\end{exercise}

\begin{exercise}
\label{ex:bilinear}

Let $\Omega=(0,1)$.  
Show that 
\[  
a(u, v) = \int_\Omega  u \, v \, dx 
\]
is a bilinear form. 
\end{exercise}


\begin{exercise}
\label{ex:poincare2}

Let $\Omega=(0,1)$.  
Show that 
\[  
a(u, v) = \int_\Omega \nabla u \cdot \nabla v \, dx 
\]
forms an inner product on $H^1_0(\Omega)$. 
\end{exercise}

\begin{exercise}
\label{ex:poincare3}

Let $\Omega=(0,1)$.  
Show that 
\[  
a(u, v) = \int_\Omega k \nabla u \cdot \nabla v \, dx 
\]
forms an inner product on $H^1_0(\Omega)$ given that $k \in \mathbb{R}^{n \times n}$ is stricktly positive.  
\end{exercise}



