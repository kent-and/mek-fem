\chapter{Alternative Formulations}


\section{Introduction}

Finite element methods are most commonly based on the the weak formulation as discussed in 
the previous chapters. However, several other formulations are possible.  
For instance, mixed methods have the advantage of mass conservation at the expense of larger linear systems.  
Furthermore, the introduction  of physics-informed networks are typically based on least square formulations.  
In this chapter we will therefore discuss different ways of formulating PDE problems. 

\section{The primal formulation}
The formulation introduced in Chapter \ref{ch:elliptic2} is often called the \emph{primal formulation}
and is equivalent with the weak formulation of the strong form.  
Let us briefly review the formulation. For simplicity of notation we only discuss elliptic problems 
with pure Dirichlet boundary conditions here.  
The \emph{strong formulation} is: Find $u_S \in C^2(\Omega)$, given $f \in C^0(\Omega)$, such that  
\begin{align}
\label{ch:alt:strong1}
-\Delta u_S &= f \quad \mbox{ in } \Omega, \\ 
\label{ch:alt:strong2}
 u_S &= g \quad \mbox{ on } \partial \Omega .  
\end{align}
The corresponding \emph{weak formulation}: Find $u_W \in H^1_g(\Omega)$, given  $f \in H^{-1}(\Omega)$, such 
that 
\begin{align}
\label{ch:alt:weak:primal}
\int_\Omega \nabla u_W \cdot \nabla v \, dx = \int_\Omega f v \, dx, \quad \forall v\in H^1_0(\Omega).       
\end{align}
The weak formulation  
follows directly from Gauss-Green´s lemma, as we saw in Chapter \ref{ch:elliptic2}. 
We will refer to this formulation as the \emph{primal formulation}. 

The weak formulation may be derived as the minimizer of an energy functional. To see this, let us  
consider the energy functional: 
\begin{align}
E (v, f) = \int_\Omega \frac{1}{2} (\nabla v)^2  - f v \, dx.   
\end{align}
The solution $u_W$ of \eqref{ch:alt:weak:primal} can equivalently be written as the minimizer of this energy functional. That is, 
\begin{align}
u_W = \argmin_{v \in H^1_g(\Omega) } E (v, f) .   
\end{align}
To see the relationship between the two formulations we need to be able to differentiate the functional $E$ with respect 
to functions $v$. Assume therefore that $H^1_g(\Omega)$ can be spanned by the basis $\{\psi_i\}^\infty_{i=1}$.
We remark that the mathematical term for spaces that can be spanned in terms of a countable basis is that 
the space is separable. It is known that $H^1_g(\Omega)$ is separable. Then 
any function $v$ in $H^1_g(\Omega)$ can be expressed as $v=\sum_i v_i \psi_i$.  We notice directly that
\begin{align} 
\frac{\partial v }{\partial v_i } = \psi_i .  
\end{align} 
From basic calculus we know that
if $u_W$ is a minimizer of $E$ then   $\frac{\partial E}{\partial u_W} = 0 $.
Considering now  $u_W=\sum_i u^W_i \psi_i$ this means that   
\begin{align} 
\frac{\partial E }{\partial u^W_i } = \int_\Omega \nabla u_W \cdot \nabla \psi_i - f \psi_i dx  = 0 .    
\end{align} 
which is exactly weak formulation, except the test functions are here termed $\psi_i$. 
We remark that there are more general and abstract ways of 
differentiating functionals such as the 
Gateaux derivative or the Fréchet derivative,  but these are not required in our setting. 

\section{Least square formulation of primal formulation}

An alternative functional, based on a least square formulation, is: 
be  
\begin{align}
E_{LS}(v, f) = \int_\Omega (-\Delta v - f)^2 \, dx .  
\end{align}
Clearly, if $u_S$ is the strong solution of \eqref{ch:alt:strong1}-\eqref{ch:alt:strong2} then $u_S$ would minimize $E_{LS}$ since
$E_{LS}(u_s, f) = 0$.    
It is, however, also clear that if $u_w$ is the weak solution of \ref{ch:alt:weak:primal}
without any additional regularity, then $E_{LS}(u_w, f) = \infty$ since $u_W$ only permits one spatial derivative.     
Hence, let us therefore assume higher regularity, ie assume that the solution $u_{LS} \in H^2_g(\Omega)$. That
is the, least square minimization problem becomes: 
\begin{align}
u_{LS} = \argmin_{v \in H^2_g(\Omega)} E_{LS} (v, f) .   
\end{align}
As before we note that $H^2_g(\Omega)$ with the basis $\{\psi_i\}$   
such that $u_{LS} = \sum_i  u^{LS}_i \psi_i$ .  
Then as before, we may differentiate and obtain a linear system of equations that determine the solution. In detail,  
\begin{align} 
\label{weak:primal:ls}
\frac{\partial E_{LS}}{\partial u^{LS}_i} = \int_\Omega \frac{\partial}{\partial u^{LS}_i} (-\Delta u_{LS} - f)^2 \, dx  = \\  
2 \int_\Omega (-\Delta u_{LS} - f) (-\Delta \psi_i) \, dx = 0 .   
\end{align} 
We remark here that this problem is clearly different from \eqref{ch:alt:weak:primal} in the sense that there is a higher 
number of spatial derivatives involved. 

In order to understand the problem  \eqref{weak:primal:ls} we derive a strong formulation of the correponsing
problem. That means that all spatial derivatives must be transfered from the test function to the trial function. 
By applying Gauss-Green´s lemma twice, we arrive at the corresponding strong formulation
of the primal form of the least square formulation:
Find $u_{LS}$ such that     
\begin{align}
\label{strong:primal:ls1}
\Delta^2 u_{LS} &= -\Delta f \quad \mbox{ in } \Omega, \\ 
\label{strong:primal:ls2}
 u_{LS} &= g \quad \mbox{ on } \partial \Omega,   \\ 
\label{strong:primal:ls3}
 -\Delta u_{LS} &= f \quad \mbox{ on } \partial \Omega .  
\end{align}
We notice that the formulation requires more regularity than the standard weak formulation of the primal problem. In fact 
\eqref{strong:primal:ls1} reguires $u \in C^4(\Omega)$ whereas   
\eqref{weak:primal:ls} is naturally posed in $H^2_g(\Omega)$.  
We also remark that since \eqref{strong:primal:ls1} is fourth order problem then two 
boundary conditions \eqref{strong:primal:ls2} and \eqref{strong:primal:ls3} are required.  
The condition \eqref{strong:primal:ls2} was present also in 
\eqref{ch:alt:strong1}, but here also  \eqref{strong:primal:ls3} appears as a condition through 
the application of Green´s lemma. While this condition appears natural, it is simply an extention
such that underlying PDE is valid also on the boundary, in general we cannot assume that the PDE is 
valid on the boundary.  

\section{The Mixed formulation}

The mixed formulation of \eqref{ch:alt:strong1} reduced the requirement in terms of
differentiability by splitting the second order equation into two first order equations. 
That is, we notice that
\begin{equation} 
\label{primal:mixed}
-\Delta u = f 
\end{equation} 
may be alternatively written as 
\begin{align}
\label{elliptic:mixed:1}
v = \nabla u, \\ 
\label{elliptic:mixed:2}
\nabla \cdot v = f . 
\end{align}
It is easily verified,  by inserting \eqref{elliptic:mixed:1} into \eqref{elliptic:mixed:2} 
that we obtain \eqref{primal:mixed} directly. Here, we notice that $v = \nabla u$ is a vector field since $u$ is a scalar field.  

The weak formulation of \eqref{elliptic:mixed:1}--\eqref{elliptic:mixed:2} is interesting as
it becomes: 
Find $v \in H(\mbox{div}, \Omega), u \in L^2(\Omega)$ such that  
\begin{align}
\label{mixed:darcy:1}
\int_\Omega v \cdot \psi \, \dx + \int_\Omega u \, \nabla \cdot \, \psi \, dx &= \int_{\partial \Omega} g \, \psi \cdot n \, ds, \quad \psi \in H(\mbox{div}, \Omega), \\  
\label{mixed:darcy:2}
\int_\Omega \nabla \cdot v \, \xi \, dx &= \int_{\Omega} f \, \xi \, dx, \quad  u \in L^2(\Omega) .  
\end{align}

The problem is a saddle-point problem and the Brezzi conditions, as discussed in Chapter \ref{chap-stokes} are
required for a well-posed problem. We notice, however, that the Brezzi conditions are   
in $H(\mbox{div}, \Omega) \times L^2(\Omega)$ rather than in $H^1(\Omega) \times L^2(\Omega)$ as it
was for the Stokes problem.  The Brezzi conditions, and most notably the inf-sup condition 
is well-known for a number of finite elements such as for example Raviart-Thomas and Brezzi-Douglas-Marini.   

We notice that through Green´s lemma the derivative of $u$ in \eqref{elliptic:mixed:1} is
moved to the test function $\psi$ and hence the regularity requirement is signficantly reduced in the sense
that only $v \in H(\mbox{div}, \Omega)$ is required. We further notice that the Dirichlet condition $u = g$ in 
in the strong formulation and weak formulation of the primal problem now appears naturally as a boundary integral.   
In fact, a Dirichlet condition for the primal problem turns into a Neumann condition for the mixed problem, while 
a Neumann condition for the primal problem becomes a Dirichlet condition for the mixed problem. 

Also this problem may be phrased as a minimization problem, although a constrained minimization problem. 
Let the energy be 
\[
E_{M}(w) = \int_\omega \frac{1}{2} w^2 \, dx  
\]
The the solution is  
\begin{align}
v = \argmin_{w \in H(\mbox{div}, \Omega) \mbox{ such that }  \nabla \cdot w = f  } E_{M} (w) .   
\end{align}
The constraint $\nabla \cdot w = f$ may be enforced using a Lagrange multiplier. Then the functional
becomes  
\[
L(w,\lambda, f) = \int_\Omega v^2 \, dx + \int_\Omega (\nabla \cdot v - g) \lambda  \, dx - \int_{\partial \Omega} (w\cdot n) \,  g)  \, ds   
\]
and 
\begin{align}
v, u = \argmin_{w} \argmax_{\lambda} L(v, \lambda, f) .   
\end{align}
Letting  $v = \sum_i v_i \psi_i$    and $u = \sum_j u_i \xi_j$ we may differentiate $L$ to find the 
linear system of equations that the extreme points must satisify.     
The system becomes:  
\begin{align}
 \frac{\partial L (v, u, f)}{ \partial v_i} =  \int_\Omega v \cdot \psi_i \, dx + \int_\Omega (\nabla \cdot \psi_i ) u \, dx  -  
 \int_{\partial \Omega} (w\cdot n) \,  g)  \, ds = 0, \\    
 \frac{\partial L (v, u, f)}{ \partial u_j} = ( \int_\Omega \nabla \cdot v \,   \xi_j \, dx  = 0 .    
\end{align}
which is recognized as 
\eqref{mixed:darcy:1}--\eqref{mixed:darcy:2} with test functions $\psi$ and $\xi$ replaced by  $\psi_i$ and $\xi_j$, respectively.  


\section{The Least Square formulation of the Mixed Problem}
The least square formulation of the primal problem had the disadvantage of requiring high regularity whereas
the mixed formulations requires special elements satisfying the Brezzi conditions. Hence, how about
creating a least square formulation based on the mixed fomrulation. The natural cost functional is then 
a slight reformulation of the Lagrangian, namely (ignoring the boundary conditions for now)    
\[
E_{LSM}(w,\lambda, f) = \int_\Omega (w-\nabla \lambda)^2 \, dx + \int_\Omega (\nabla \cdot w - f)^2   \, dx    
\]
The solution $v, u$ becomes  
\[
v, u = \argmin_{v, \lambda} E_{LSM}(w,\lambda, f) 
\]
The linear system can again be written as a system  
\begin{align}
\label{mixed:ls:1}
 \frac{\partial E_{LMS} (v, u, f)}{ \partial v_i} &=  \int_\Omega (v - \nabla u) \cdot \psi_i \, dx + \int_\Omega \nabla \cdot v \,  \nabla \cdot \psi_i  \, dx  \\   
\label{mixed:ls:2}
 \int_{\partial \Omega} (w\cdot n) \,  g)  \, ds = 0, \\    
 \frac{\partial E_{LMS} (v, u, f)}{ \partial u_j} &= ( \int_\Omega (v - \nabla  u)\cdot(-\nabla \xi_j) \, dx  = 0 .    
\end{align}
which is recognized as 
\eqref{mixed:darcy:1}--\eqref{mixed:darcy:2} with test functions $\psi$ and $\xi$ replaced by  $\psi_i$ and $\xi_j$, respectively.  

\section{Least Square formulation of the primal form and Dirichlet boundary conditions}

One of the interesting features of physics-informed neural networks is that it has popularized 
least square formulations of both the PDE and the boundary conditions. Hence, as
an alternative formulation of \eqref{ch:alt:strong1} -- \eqref{ch:alt:strong2}   
\begin{align}
E_{LSB}(v, f) = \int_\Omega (-\Delta v - f)^2 \, dx + \alpha^2 \int_{\partial \Omega} (v - g)^2 \, ds .  
\end{align}
Here, $\alpha$ is a parameter that enables a different weighting of the boundary term.  

As earlier, the solution is defined as the minimizer: 
\begin{align}
\label{ch:alt:lsb}
u_{LSB} = \argmin_{v \in H^2(\Omega)} E_{LSB} (v, f) .   
\end{align}
which we may differentiate to find the corresponding weak formulation. 

After arriving at the weak form, we use Green´s lemma to arrive at the 
corresponding strong problem. The problem is a bi-harmonic problem that reads: 
Find $u$ such that 
\newcommand{\partialn}{\partial_{\mathbf{n}}}
\begin{align}
\label{L1strong1Alt}
\Delta^2 u &= -\Delta f, \quad &\mbox{ in } \ \Omega, \\
\label{L1strong2Alt}
         \alpha^2 u - \partialn \Delta u &= \alpha^2 g + \partialn f, \quad &\mbox{ on } \ \partial \Omega, \\
\label{L1strong3Alt}
         -\Delta u &= f, \quad &\mbox{ on } \ \partial \Omega. 
\end{align}
The well-posedness of the problem is complicated by the fact that the solution is sought in $H^2$ 
as in with the least square formulation of the primal problem.  
However, for $u \in H^2(\Omega)$ we have  $T u \|_{\partial \Omega} \in H^{3/2}(\partial \Omega)$, but
we are not able to enforce the boundary conditions in the strong $H^{3/2}(\partial \Omega)$ sense
only in  $L^{2}(\partial \Omega)$. Therefore, since the boundary conditions are only enforced in this
weak sense, there is potentially a loss of regularity near the boundary. The 
loss of regularity may be partially resolved by weakening the coersivity~\cite{zeinhofer2024unified} 
or by strengthening the regularity at the boundary~\cite{bonito2025convergence}.  





\section{Exercises}

\begin{exercise}
Calculate the \emph{strong} form of \eqref{mixed:ls:1}  -- \eqref{mixed:ls:2}.  
\end{exercise}

\begin{exercise}
Calculate the \emph{weak}  form  of 
\eqref{ch:alt:lsb} and show that the strong form is
\eqref{L1strong1Alt}  -- \eqref{mixed:ls:2}.  
\end{exercise}
 
 
