\chapter{Alternative Formulations}


\section{Introduction}

Most commonly, the basis for finite element methods has been the weak formulation as discussed in 
the previous chapters. But, with the introduction of neural networks and physics-informed networks, 
alternative formulations based on least square method principles have been made popular. Such formulations
have also been studied earlier, but has in general received little attention as compared with stdandard weak formulations.   
In this chapter we will therefore discuss different ways of formulating PDE problems. 
Let us for simplicity discuss elliptic problems with Dirichlet boundary conditions of the 
form: Find $u \in C^2(\Omega)$ given $f \in C^0(\Omega)$ of the form
\begin{align}
-\Delta u &= f \quad \mbox{ in } \Omega, \\ 
 u &= g \quad \mbox{ on } \Omega .  
\end{align}
The weak formulation: Find $u \in H^1_g(\Omega)$, given  $f \in H^{-1}(\Omega)$ such 
that 
\begin{align}
\label{weak:primal}
\int_\Omega \nabla u \cdot \nabla v \, dx = \int_\Omega f v \, dx, \quad \forall v\in H^1_0(\Omega),       
\end{align}
follows directly from Gauss-Green´s lemma, as we saw in Chapter \ref{ch:elliptic2}. 
We will refer to this formulation as the \emph{primal formulation}. 

The weak formulation may alternatively be derived from energy considerations. That is, 
consider the energy functional, 
\begin{align}
E (v, f) = \int_\Omega \frac{1}{2} (\nabla v)^2  - f v \, dx.   
\end{align}
The solution of \eqref{weak:primal} can equivalently be written as the minimizer of this energy functional. That is, 
\begin{align}
u = \argmin_{v \in H^1_g(\Omega) } E (v, f) .   
\end{align}
\kent{use u, v, $w_i$ such that both u and v can be expressed as a linear combination of $w_i$}
To see the relationship between the two formulations we need to be able to differentiate the functional $E$ with respect 
to functions $v$. Assume therefore that $H^1_g(\Omega)$ can be spanned by a basis $\{v_i\}^\infty_{i=1}$. Then 
a function can be expressed as $u=\sum_i u_i v_i$ and we notice directly that
\begin{align} 
\frac{\partial u }{\partial u_i } = v_i .  
\end{align} 
Hence, 
\begin{align} 
\frac{\partial E }{\partial u_i } = \int_\Omega \nabla u \cdot \nabla v_i - f v_i dx  .  
\end{align} 
Furthermore, clearly $E(v, f)$ is minimized when the derivative is zero, i.e. 
\begin{align} 
\frac{\partial E }{\partial u_i } = \int_\Omega \nabla u \cdot \nabla v_i - f v_i dx = 0  .  
\end{align} 
which corresponds to the weak formulation. 

\section{Least square formulation of primal formulation}

An alternative "energy"-like functional, based on least square principles that could be minimized would  
be  
\begin{align}
E_{LS} = \|-\Delta u - f\|_{L^2(\Omega)}
\end{align}
Again, we seek to find the minimizer
\begin{align}
u = \argmin_{v \in ? } E_{LS} (v, f) .   
\end{align}
We differentiate in order to find that 
\begin{align} 
\label{weak:primal:ls}
\frac{\partial E_{LS}}{\partial u_i} = \int_\Omega \frac{\partial}{\partial u_i} (-\Delta u - f)^2 \, dx  = \\  
2 \int_\Omega (-\Delta u - f) (-\Delta v_i) \, dx 
\end{align} 
By applying Gauss-Green´s lemma again we may again derive a corresponding strong formulation:
Find $u$ such that     
\begin{align}
\label{strong:primal:ls}
\Delta^2 u &= -\Delta f \quad \mbox{ in } \Omega, \\ 
 u &= g \quad \mbox{ on } \partial \Omega,   \\ 
 -\Delta u &= f \quad \mbox{ on } \partial \Omega .  
\end{align}

We refer \eqref{weak:primal:ls} and \eqref{strong:primal:ls} as the weak and strong formulation of the least square formulation of the
primal problem.  
We notice that both formulations requires more regularity than the standard weak formulation of the primal problem. In fact 
\eqref{strong:primal:ls} reguires $u \in C^4(\Omega)$ whereas   
\eqref{weak:primal:ls} is naturally posed in $H^2_g(\Omega)$.  

\section{The Mixed formulation}

To reduce the number of derivatives on the solution, a common 
trick is to splitt the original second order equation into 
two equations of first order. That is, we notice that
\begin{equation} 
\label{primal:mixed}
-\Delta u = f 
\end{equation} 
may be alternatively written as 
\begin{align}
\label{elliptic:mixed:1}
\psi = \nabla u, \\ 
\label{elliptic:mixed:2}
\nabla \cdot \psi = f . 
\end{align}
That is, inserting \eqref{elliptic:mixed:1} into the \eqref{elliptic:mixed:2} directly gives 
\eqref{primal:mixed}. Here, we notice that $\psi$ is a vector field while $u$ is a scalar field.  

Also this problem may be phrased as a minimization problem, although a constrained minimization problem. 
Let the energy be 
\[
E_{M}(\psi) = \int_\omega \frac{1}{2} \psi^2 \, dx  
\]
The the solution is  
\begin{align}
u = \argmin_{\nabla \cdot \psi = f  } E_{M} (v) .   
\end{align}
The constraint $\nabla \cdot v = f$ may be enforced using a Lagrange multiplier. Then the functional
becomes  
\[
L(v,\lambda, f) = \int_\Omega v^2 \, dx + \int_\Omega (\nabla \cdot v = g) \lambda  \, dx   
\]
and 
\begin{align}
u, \mu = \argmin_{v} \argmax_{\lambda} L(v, \lambda, f) .   
\end{align}
Again, we may differentiate (with respect to both $v$ and the Lagrange multiplier $\lambda$ to find the extremal points of $L$ 
as 
\begin{align}
 \frac{\partial L (u, \lambda)}{ \partial v} =  \int_\Omega v^2 \, dx + \int_\Omega (\nabla \cdot v = g) \lambda  \, dx   
\end{align}


 
