\documentclass[a4paper,11pt]{amsart}
\usepackage{graphicx,color}
\usepackage{codehighlight}


\long\def\COMMENT#1{\par\vbox{\hrule\vskip1pt \hrule width.125in height0ex #1\vskip1pt\hrule}}

\newcommand{\norm}[1]{ |\!|\!| #1 |\!|\!|}



\newtheorem{theorem}{Theorem}[section]
\newtheorem{example}{Example}[section]
\newtheorem{exercise}{Exercise}[section]
\newtheorem{remark}{Remark}[section]
%\newtheorem{proof}{Proof}[section]


\title{Discretization of the Stokes problem}
\begin{document}
\maketitle

\section{Introduction}
The Stokes problem describes the flow of a slowly moving viscous incompressible Newtonian
fluid.  Let the fluid domain be denoted $\Omega$. We assume that $\Omega$ is a bounded domain in $\mathbb{R}^n$ with a smooth boundary. Furthermore, let $u : \Omega \rightarrow \mathbb{R}^n$ be
the fluid velocity and $p:\Omega \rightarrow \mathbb{R}$ be the fluid pressure.
The strong form of the Stokes problem can then be written as
\begin{eqnarray}
-\Delta u + \nabla p &=& f, \mbox{ in } \Omega,   \\
\nabla \cdot u &=& 0, \mbox{ in } \Omega, \\
u &=& g, \mbox{ on } \partial \Omega_D, \\
\frac{\partial u}{\partial n} - p n  &=& h, \mbox{ on } \partial \Omega_N.
\end{eqnarray}
Here, $f$ is the body force, $\partial \Omega_D$ is the Dirichlet
boundary, while $\partial \Omega_N$ is the Neumann
boundary. Furthermore, $g$ is the prescribed fluid velocity on the
Dirichlet boundary, and $h$ is the surface force or stress on the
Neumann boundary. These boundary condition leads to a well-posed
problem provided that neither the Dirichlet nor Neumann boundaries are
empty.  In case of only Dirichlet conditions the pressure is only
determined up to a constant, while only Neumann conditions leads to
the velocity only being determined up to a constant.

These equations are simplifications of the Navier-Stokes equations for
very slowly moving flow.  In contrast to elliptic equations, many
discretizations of this problem will lead to instabilities.  These
instabilities are particularly visible as non-physical oscillations in
the pressure. The following example illustrate such oscillations.

\begin{example}{Poiseuille flow} \\
One of the most common examples of flow problems that can be solved
analytically is Poiseuille flow. It describes flow in a straight
channel (or cylinder in 3D).  The analytical solution is $u=(y\,
(1-y), 0)$ and $p = 1-x$.  Since the solution is know, this flow
problem is particularly useful for verifying that the code or
numerical method. We therefore begin by discretizing the problem in
the simplest way possible, i.e., linear continuous/Lagrange elements
for both velocity and pressure. The results is shown Figure
\ref{fig:stokes1}. Clearly, the velocity is approximated satisfactory,
but the pressure oscillate widely and is nowhere near the actual
solution.
\begin{figure}
\begin{center}
\includegraphics[width=6cm]{../pics/stokes_velocity.png}
\includegraphics[width=6cm]{../pics/stokes_pressure_instabilities.png}
\caption{Poiseuille flow solution obtained with linear continuous elements for
both velocity and pressure. The left figure shows the (well-represented) velocity while the right shows
the pressure (with the wild oscillations).}
\label{fig:stokes1}
\end{center}
\end{figure}

% inf56xx-2013/code/stokes/stokes_p1p1.py
\begin{python}
from dolfin import *

def u_boundary(x):
  return x[0] < DOLFIN_EPS or x[1] > 1.0 - DOLFIN_EPS or x[1] < DOLFIN_EPS

def p_boundary(x):
  return  x[0] > 1.0 - DOLFIN_EPS

mesh = UnitSquare(40,40)
V = VectorFunctionSpace(mesh, "Lagrange", 1)
Q = FunctionSpace(mesh, "Lagrange", 1)
#Q = FunctionSpace(mesh, "DG", 0)
W = MixedFunctionSpace([V, Q])

u, p = TrialFunctions(W)
v, q = TestFunctions(W)

f = Constant([0,0])

u_analytical = Expression(["x[1]*(1-x[1])", "0.0"])
p_analytical = Expression("-2+2*x[0]")

bc_u = DirichletBC(W.sub(0), u_analytical, u_boundary)
bc = [bc_u]

a = inner(grad(u), grad(v))*dx + div(u)*q*dx + div(v)*p*dx
L = inner(f, v)*dx

UP = Function(W)
A, b = assemble_system(a, L, bc)
solve(A, UP.vector(), b, "lu")

U, P = UP.split()

plot(U, title="Numerical velocity")
plot(P, title="Numerical pressure")

U_analytical = project(u_analytical, V)
P_analytical = project(p_analytical, Q)

plot(U_analytical, title="Analytical velocity")
plot(P_analytical, title="Analytical pressure")

interactive()
\end{python}



However, when using the second order continuous elements for the velocity and
first order continuous elements for the pressure, we obtain the perfect solution
shown in Figure \ref{fig:stokes2}.
\begin{figure}
\begin{center}
\includegraphics[width=6cm]{../pics/stokes_velocity.png}
\includegraphics[width=6cm]{../pics/stokes_pressure.png}
\caption{Poiseuille flow solution obtained with quadratic continuous elements for the
velocity and linear continuous elements for the pressure. The left figure shows the velocity while the right shows
the pressure. Both the velocity and the pressure are correct.}
\label{fig:stokes2}
\end{center}
\end{figure}
\end{example}

The previous example demonstrates that discretizations of the Stokes problem may lead
to, in particular, strange instabilities in the pressure. In this chapter we will describe why this
happens and several strategies to circumvent this behaviour.

Let us first start with a weak formulation of Stokes problem:
Find $u\in H^1_{D,g}$ and $p\in L^2$.
\begin{eqnarray*}
a(u,v) + b(p,v) &=& f(v), \quad v\in H_{D,0}^1\\
b(q,u) &=& 0,\quad q\in L^2,
\end{eqnarray*}
where
\begin{eqnarray*}
a(u,v) &=& \int\nabla u : \nabla v\ dx, \\
b(p,v) &=& \int p \, \nabla \cdot v\ dx, \\
f(v) &=& \int f\, v\ dx + \int_{\Omega_N} h \, v \, ds  .
\end{eqnarray*}
Here
$H^1_{D,g}$ contains functions  in $H^1$ with trace $g$ on $\partial \Omega_D$.
To obtain symmetry we have substituted $\hat{p} = - p$ for the pressure
and is referint to $\hat{p}$ as p.

As before the standard finite element formulation follows directly from the weak formulation:
Find $u_h\in V_{g,h}$ and $p_h\in Q_h$ such that
\begin{eqnarray}
\label{eq:stdGalerkin1}
a(u_h, v_h ) + b(p_h, v_h) &=& f(v_h),\quad \forall v_h\in V_{0,h}, \\
\label{eq:stdGalerkin2}
b(q_h, v_H) &=& 0,\quad \forall q_h \in Q_h .
\end{eqnarray}
Letting $u_h=\sum_{i=1}^n u_i N_i$, $p_h=\sum_{i=1}^n p_i L_i$, $v_h=N_j$, and $q_h=L_j$
we obtain a linear system on the form
\begin{equation}
\left[ %matrix
	\begin{array}{cc}
	A & B\\
	B^T & 0
	\end{array}
\right]
\left[ %vector
	\begin{array}{c}
	v\\
	p
	\end{array}
\right]
=
\left[ %vector
	\begin{array}{c}
	f\\
	0
	\end{array}
\right]
\label{la:stokes}
\end{equation}
Here
\begin{eqnarray}
A_{ij} = a(N_i,N_j) &=& \int\nabla N_i \nabla N_j \ dx, \\
B_{ij} = b(L_i,N_j) &=& \int\nabla L_i \, N_j \ dx.
\end{eqnarray}
Hence, $A$ is $n\times n$, while $B$ is $n\times m$,
where $n$ is the number of degrees of freedom for the velocity field, while
$m$ is the number of degrees of freedom for the pressure.

Is the system \eqref{la:stokes} invertible?  For the moment, we assume that the submatrix $A$ is invertible. This is typically the case for
Stokes problem. We may then perform blockwise Gauss elimination:
That is,
we multiply the first equation with $A^{-1}$ to obtain
\[v = A^{-1}f - A^{-1}Bp\]
Then, we then insert $v$ in the second equation to get
\[0 = B^Tv = B^TA^{-1}f - B^TA^{-1}Bp\]
i.e we have removed $v$ and obtained an equation only involving $p$:
\[B^TA^{-1}Bp = B^TA^{-1}f\]
This equation is often called the pressure Schur complement. The question is then reduced to whether $B^TA^{-1}B$ is invertible.
Consider the follwing two situations:

\begin{tabular}[h]{lcr}
\setlength{\unitlength}{0.090in}
\begin{picture}(15,15)
\thinlines
\put(1,0){\line(0,3){3}}
\put(0.7, 0){\line(1,0){0.7}}
\put(0.7, 3){\line(1,0){0.7}}
\put(-1,1.5){$m$}
\thicklines
\put(2,0){\framebox(9,3){$B^T$}}

\thinlines
\put(1.9,13.6){\line(9,0){9}}
\put(1.9, 13.8){\line(0,-1){0.5}}
\put(11, 13.8){\line(0,-1){0.5}}
\put(5.5,14){$n$}
\thicklines
\put(2,4){\framebox(9,9){$A$}}

\put(12,0){\framebox(3,3){$0$}}
\put(12,4){\framebox(3,9){$B$}}
\end{picture}
\vspace{\unitlength}

\qquad   v.s \qquad

\setlength{\unitlength}{0.090in}
\begin{picture}(15,15)
\thinlines
\put(1,0){\line(0,9){9}}
\put(0.7, 0){\line(1,0){0.7}}
\put(0.7, 9){\line(1,0){0.7}}
\put(-1,4.5){$m$}
\thicklines
\put(2,0){\framebox(3,9){$B^T$}}

\thinlines
\put(1.9,13.6){\line(3,0){3}}
\put(1.9, 13.8){\line(0,-1){0.5}}
\put(5, 13.8){\line(0,-1){0.5}}
\put(2.5,14){$n$}
\thicklines
\put(2,10){\framebox(3,3){$A$}}

\put(6,0){\framebox(9,9){$0$}}
\put(6,10){\framebox(9,3){$B$}}
\end{picture}
\vspace{\unitlength}
\end{tabular}

Clearly, the right most figure is not invertible since $n \ll  m$ and
the $0$ in the lower right corner dominates. For the left figure on might expect that the
matrix is non-singular since $n \gg m$, but it will depend on $A$ and $B$. We have already
assumed that $A$ is invertible, and we therefore ignore $A^{-1}$ in $B^TA^{-1}B$.
The question is then whether $B^TB$ is invertible.

\setlength{\unitlength}{0.090in}
\begin{picture}(15,15)
\thicklines
\put(0,6){\framebox(9,3){$B^T$}}
\put(10,0){\framebox(3,9){$B$}}
\put(14,7){$=$}

\thinlines
\put(16,9.6){\line(3,0){3}}
\put(15.9, 9.8){\line(0,-1){0.6}}
\put(19, 9.8){\line(0,-1){0.6}}
\put(15.5,11){\small{$m\!\times\!m$}}
\thicklines
\put(16,6){\framebox(3,3){}}
\label{BBinvertible}
\end{picture}
\vspace{\unitlength}

As illustrated above, $B^T B$ will be a relatively small matrix compared to
$B$ and $A$ as long as $n  \gg m$. Therefore, $B^T B$  may therefore be non-singular.
To ensure that $B^T B$ is invertible, it is necessary that
\[\operatorname{kernel}(B^T)=0,\ \textrm{where}\ B^T \textrm{is}\ n\times m\]
An equvialent statement is that
\begin{equation}
\max_v\ (v,B^Tp) > 0\quad \forall\ p .
\label{max1}
\end{equation}
Alternatively,
\begin{equation}
\max_v\ \frac{(v,B^Tp)}{\|v\| } \ge \beta \|p\| \quad \forall\ p.
\label{max2}
\end{equation}
Here, $\beta > 0$. We remark that \eqref{max1} and \eqref{max2} are equivalent for a finite dimensional matrix.
However, in the infinite dimentional setting of PDEs \eqref{max1} and \eqref{max2} are different.
Inequality \eqref{max1} allow $(v, B^T p)$ to approach zero, while \eqref{max2} requires a lower bound.
For the Stokes problem, the corresponding condition is crucial:
\begin{equation}
\sup_{v\in H^1_{D,g}}\ \frac{(p, \nabla\cdot u) }{\|u\|_1\ } \geq \beta \|p\|_0 > 0, \quad  \forall p\in L_2
\label{infsup:cont}
\end{equation}

Similarly, to obtain order optimal convergence rates, that is
\[\|u-u_h\|_1 + \|p-p_h\|_0 \leq Ch^k\|u\|_{k+1} + Dh^{\ell+1}\|p\|_{\ell+1}\]
where $k$ and $\ell$ are the ploynomial degree of the velocity and the pressure, respectively,
the celebrated \emph{Babuska-Brezzi condition} has to be satisfied:
\begin{equation}
\sup_{v\in V_{h,g}}\ \frac{(p, \nabla\cdot u) }{\|u\|_1\ } \geq \beta \|p\|_0 > 0, \quad  \forall p\in Q_h
\label{infsup:disk}
\end{equation}
We remark that the discrete condition \eqref{infsup:disk} does not follow from \eqref{infsup:cont}.
In fact, it is has been a major challenge in numerical analysis
to determine which finite element pairs $V_h$ and $Q_h$ that meet this condition.

\begin{remark}
For saddle point problems on the form \eqref{eq:stdGalerkin1}-\eqref{eq:stdGalerkin2} four conditions
have to be satisfied in order to have a well-posed problem: \\
Boundedness of $a$:
\begin{equation}
\label{remark1}
a(u_h, v_h) \le C_1 \|u_h\|_{V_h} \|v_h\|_{V_h}, \quad \forall u_h, v_h \in V_h,
\end{equation}
and boundedness of $b$:
\begin{equation}
\label{remark2}
b(u_h, q_h) \le C_2 \|u_h\|_{V_h} \|q_h\|_{Q_h},  \quad \forall u_h \in V_h,  q_h \in Q_h,
\end{equation}
Coersivity of $a$:
\begin{equation}
\label{remark3}
a(u_h, u_h) \ge C_3 \|u_h\|^2_{V_h} , \quad \forall u_h \in V_h,
\end{equation}
and "coersivity" of $b$:
\begin{equation}
\label{remark4}
\sup_{u_h\in V_h} \frac{b(u_h, q_h)}{\|u_h\|_{V_h}} \ge C_4 \|q_h\|_{Q_h} , \quad \forall q_h \in Q_h.
\end{equation}
For the Stokes problem, \eqref{remark1}-\eqref{remark3} are easily verified, while \eqref{remark4} often
is remarkably difficult unless the elements are designed to meet this condition.
\end{remark}



\section{Examples of elements}
\subsection{The Taylor-Hoood element}
The Taylor-Hood elements are quadratic for the velocity and linear for pressure
\begin{eqnarray*}
v:\ N_i &=& a_i^v + b_i^vx + c_i^vy + d_i^vxy + e_i^vx^2 + f_i^vy^2 \\
p:\ L_i &=& a_i^p + b_i^px + c_i^py
\end{eqnarray*}
and are continuous across elements

\begin{figure}[h]
\begin{center}
\includegraphics[width=4cm]{../pics/TH.pdf}
\includegraphics[width=4cm]{../pics/linear.pdf}
\caption{Taylor-Hood: quadratic element for $v$ and linear element for $p$}
\end{center}
\end{figure}

For the Taylor-Hood element we have the following error estimate:
\[\|u-u_h\|_1 + \|p-p_h\|_0 \leq Ch^2 (\|u\|_{3} + \|p\|_{2})\]

\subsection{The Crouzeix-Raviart element}
This element is linear in velocity and constant in pressure
\begin{eqnarray*}
v:\ N_i &=& a_i^v + b_i^vx + c_i^vy\\
p:\ L_i &=& a_i^p
\end{eqnarray*}
The $v$ element is continuous \emph{only} in the mid-point of each side (see \ref{fig:CR_CR}), and the $p$ element is discontinuous. This ensures that it satisfies the Babuska-Brezzi condition.

\begin{figure}[h]
\begin{center}
\label{fig:CR_CR}
\includegraphics[width=4cm]{../pics/CR.pdf}
\includegraphics[width=4cm]{../pics/const.pdf}
\caption{Crouzeix-Raviart: mid-point linear element for $v$ and constant element for $p$}
\end{center}
\end{figure}

For the Crouzeix-Raviart element we have the following error estimate:
\[\|u-u_h\|_1 + \|p-p_h\|_0 \leq Ch (\|u\|_{2} + \|p\|_{1})\]


\subsection{The P1-P0 element}

If we on the other hand choose to locate the nodal points on the corners in the $v$ element as shown in \ref{fig:PP_lin} (called a $\mathit{P_1-P_0}$ \emph{element}) the inf-sup condition is not satisfied and we get oscillations in the pressure term.

\begin{figure}[h]
\begin{center}
\label{fig:PP_lin}
\includegraphics[width=4cm]{../pics/linear.pdf}
\includegraphics[width=4cm]{../pics/const.pdf}
\caption{$P_1-P_0$: linear element for $v$ and constant element for $p$}
\end{center}
\end{figure}

\subsection{The Mini element}
The mini element is linear in both velocity and pressure, but the velocity element contains a qubic bubble. Notice that elements that are linear in both $v$ and $p$ will not satisfy the inf-sup condition.  Thus we add the extra bubble in $v$ to give an extra degree of freedom as depicted in \ref{fig:mini}.

\begin{figure}[h]
\begin{center}
\label{fig:mini}
\includegraphics[width=4cm]{../pics/mini.pdf}
\includegraphics[width=4cm]{../pics/linear.pdf}
\caption{Mini: linear element with bubble for $v$ and linear element for $p$}
\end{center}
\end{figure}

For the Crouzeix-Raviart element we have the following error estimate:
\[\|u-u_h\|_1 + \|p-p_h\|_0 \leq Ch (\|u\|_{2} + \|p\|_{1})\]

\section{Stabilization techniques to circumwent the Babuska-Brezzi condition}
Stabilization techniques typically replace the system:
\begin{eqnarray*}
Au + B^Tp &=& f \\
Bu &=& 0
\end{eqnarray*}
with an alternative system
\begin{eqnarray*}
Au + B^Tp &=& f \\
Bu -\epsilon Dp &=& \epsilon d ,
\end{eqnarray*}
where $\epsilon$ is properly chosen.

To see that we obtain a nonsingular system we again multiply the first
equation with $A^{-1}$ and then factorize:
\begin{eqnarray*}
u &=& A^{-1}f - A^{-1}B^Tp \\
Bu &=& BA^{-1}f - BA^{-1}B^Tp  = \epsilon d + \epsilon Dp \\
(BA^{-1}B^T + \epsilon D)p &=& BA^{-1}f - \epsilon d
\end{eqnarray*}
If $D$ is nonsingular then
$(BA^{-1}B^T + \epsilon D)$ will be is nonsingular since both $D$ and $BA^{-1}B^T$ are positive (only $D$ is positive definite however).

Factorizing for $p$ we end up with a \emph{Velocity-Schur complement}. Solving for $p$ in the second equation and inserting the expression for $p$ into the first equation we have
\begin{eqnarray*}
p &=& (-\epsilon D)^{-1}(\epsilon d - Bu)\\
&\Downarrow&\\
Au + B^T(-\epsilon D)^{-1}(\epsilon d-Bu) &=& f\\
(A + \frac{1}{\epsilon}B^TD^{-1}B)u &=& f + D^{-1}d
\end{eqnarray*}
$(A + \frac{1}{\epsilon}B^TD^{-1}B)$ is nonsingular since $A$ is nonsingular and $B^TD^{-1}B$ is positive.
%---end section Alternative Stabilization---

At least, three techniques have been proposed for stabilization. These are:
\begin{enumerate}
\item $\nabla\cdot v = \epsilon\Delta p$. Pressure stabilization. Motivated through mathematical intuition (from the convection-diffusion equation).
\item $\nabla\cdot v = -\epsilon p$. Penalty method. Typically, one uses the Velocity-Schur complement
\item $\nabla\cdot = -\epsilon\frac{\partial p}{\partial
  t}$. Artificial compressibility. A practical method as one adds the
  possibility for time stepping.
\end{enumerate}

\noindent
In other words, these techniques sets $D$ to be
\begin{enumerate}
	\item $D=A$
	\item $D=M$
	\item $D=\frac{1}{\Delta t}M$
\end{enumerate}
where $A$ is the stiffness matrix (discrete laplace operator) and $M$ is the mass matrix.


\section{Exercises}

\begin{exercise}
Show that the conditions \eqref{remark1}-\eqref{remark3} are satisfied for
$V_h = H^1_0$ and $Q_h=L_2$.
\end{exercise}
\begin{exercise}
Show that the conditions \eqref{remark1}-\eqref{remark3} are satisfied for
Taylor--Hood and Mini discretizations.
(Note that Crouzeix-Raviart is non-conforming so it is more difficult to prove these conditions
for this case.)
\end{exercise}
\begin{exercise}
Condition \eqref{remark4} is difficult to prove. However, if we assume that
$V_h = L_2$ and $Q_h=H^1_0$, you should be able to prove it. (Hint: This
is closely related to Poincare's inequality.)
\end{exercise}

\begin{exercise}
Test other finite elements for the Poiseuille flow problem. Consider $P_1-P_0$, $P_2-P_2$, $P_2-P_0$, as
well as the Mini and Crouzeix-Raviart element.
\end{exercise}

\begin{exercise}
Implement stabilization for the Poiseuille flow problem and use first order linear elements
for both velocity and pressure.
\end{exercise}

\begin{exercise}
In the previous problem the solution was a second order polynomial in the velocity and
first order in the pressure. We may therefore obtain the exact solution and it
is therefore difficult to check order of convergence for higher order methods with
this solution. In this exercise you should therefore
implement the problem $u=(sin(\pi y), cos( \pi x)$, $p=sin(2 \pi x)$, and $f = -\Delta u - \nabla p$.
Test whether the approximation is of the expected order for $P_4-P_3$, $P_4-P_2$, $P_3-P_2$, and $P_3-P_1$.

\end{exercise}

\begin{exercise}
Implement the problem $u=(sin(\pi y), cos( \pi x)$, $p=sin(2 \pi x)$, and $f = -\Delta u - \nabla p$
and determine the order of the approximation of wall shear stress.
\end{exercise}




\section{Further reading}



\end{document}
