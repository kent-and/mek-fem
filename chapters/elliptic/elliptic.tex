\chapter{Elliptic equations and the finite element method}


%\newcommand{\R}{\mathbb{R}}


\label{elliptic}
\section{Introduction}

As a starting point for the finite element method, let us
consider the mother problem of partial differential equations, 
the elliptic problem: Find the solution $u$ of the problem

\begin{eqnarray}
\label{elliptic}
-\nabla\cdot(k\nabla u)  &=& f \quad \textrm{in}\ \Omega,\\
\label{Dirichlet}
u&=& g \quad \textrm{on}\ \partial\Omega_D, \\
\label{Neumann}
\frac{\partial u}{\partial n}&=& h \quad \textrm{on}\ \partial\Omega_N . 
\end{eqnarray}
We include here both Dirichlet \eqref{Dirichlet} and Neumann \eqref{Neumann} boundary conditions
and assume $\partial \Omega = \partial \Omega_D \cup \partial \Omega_N$ 
and $\partial \Omega_D \cap \partial \Omega_N = \Null$.
 

Unusual concepts like weak or variational formulations, trial and test functions, Sobolev
spaces etc. show up in the finite element methods and many find them troublesome and strange. To motivate
these concepts we start with some "philosophical" considerations that lead to three observations which 
will be resolved by the finite element method. First, the above equation is 
the so-called strong formulation and its interpretation is (directly) that for every point in $\Omega$
the equation $ -\nabla\cdot(k\nabla u)  = f$ should be valid. Hence, $u$, $f$ are functions and
if we assume that $f$ is a continuous function then $u$ is continuous with two derivatives that are also continuous.   
More formally, $f\in C(\Omega)$ directly leads to the requirement that $u\in C^2(\Omega)$. In general 
it is however well known, and we will meet many of them in this course, that $u \not \in C^2(\Omega)$ are
solutions to \eqref{elliptic}. 

Secondly, from a linear algebra point of view, let us consider the linear system 
\begin{equation}
\label{Aub}
A u = b 
\end{equation}
where $A$ is an $\R^{N\times N}$ matrix whereas $u$, $f$ are vectors in $\R^N$ and 
$N$ is the number of points in $\Omega$ which could be infinite. Assuming for instance
that $\Omega$ is the unit square with $n$ points in both the $x-$ and $y-$direction such 
that $N=n^2$. Then the unknowns of 
\eqref{Aub} are $u_k = u(x_i, y_i)$ whereas
the equations are     
$-\nabla\cdot (k \nabla  u(x_i, y_i) = f(x_i, y_i)$ and we notice that
we have $N$ unknowns and $N$ equations leading to a linear system with 
a square and possibly non-singular matrix, however of infinite size.  

For computational methods, we need to resort to finite resolution and this brings us to our third 
issue. Triangulated domain. 

The finite element method resolves these three challenges by combining so-called weak formulations
to reduced the demands of differentiability of the solution with a structured approach of integration
adjusted to underlying meshes and polynomial approximation to achieve a versatile, practical, accurate
and efficient method. However, before we start with FEM, let us recap some fundamental results of calculus, 
Gauss-Green's lemma: 
\begin{equation}
\label{gaussgreen}
\int_\Omega 
-\nabla\cdot(k\nabla u) v \dx =  
\int_\Omega k\nabla u) \nabla v \dx  - \int_{\partial \Omega} k\frac{\partial u} {\partial n} \ds 
\end{equation}

With this lemma in mind we transform equation \eqref{elliptic} into an integral version by multiplying
\eqref{elliptic} with a test function $v$ before we integrate over the domain $\Omega$. Hence, 
we obtain  
\begin{equation}
\label{gaussgreen}
\int_\Omega -\nabla\cdot(k\nabla u) v \dx =  \int_\Omega f v \dx .  
\end{equation}
Here, $v$ plays the role the pointwise evaluation in the strong formulation. That is, 
if $v$ is the Dirac delta functions corresponding to the points in $\Omega$, i.e.,  $v_k = \delta(x_i, y_j)$
then we obtain directly the strong formulation. This formulation is often called
the collocation method and is used e.g. for the Runge-Kutta method. However, it is 
seldom used for FEM because of the high demands on the differentiability of $u$. 

The weak formulation exploit the Gauss-Green´s lemma together with the boundary conditions. That is,  
\begin{eqnarray*}
\label{gaussgreen}
\int_\Omega f v \dx =   \int_\Omega -\nabla\cdot(k\nabla u) v \dx =  \\ 
\int_\Omega k\nabla u) \nabla v \dx  - \int_{\partial \Omega} k\frac{\partial u} {\partial n} v \ds   
\end{eqnarray*}

Finally, to get to the weak formulation, we have to apply boundary conditions. For the Dirichlet
condition \eqref{Dirichlet} we already know that $u=g$. Since we already know $u$ here it is common 
to say also that $v=0$ is known. Furthermore, by applying the Neumann condition, we then obtain that  
\begin{eqnarray}
\int_{\partial \Omega} k\frac{\partial u} {\partial n} v ds =   
\int_{\partial \Omega_D} k\frac{\partial u} {\partial n} v ds +   \int_{\partial \Omega_N} k\frac{\partial u} {\partial n} v ds =   
\int_{\partial \Omega_D} k\frac{\partial u} {\partial n} 0 +   \int_{\partial \Omega_N} h v ds   
\end{eqnarray}
As such we arrive at the \emph{weak formulation} of the elliptic problem: 
Find $u$ such that 
\[
\int_\Omega k \nabla u \cdot \nabla v dx = \int_\Omega f v dx + \int_{\Omega_N} h v ds, \quad \forall v  
\]
Here, we assume as mentioned that $u=g$ and $v=0$ on $\partial \Omega_D$.  


The finite element method directly exploits the weak formulation. 
Let $u = \sum_j u_j N_j$ and $v=N_i$. 
Then 
\[
\int_\Omega k \nabla \sum_j u_j N_j  \cdot \nabla N_i dx = \int_\Omega f N_i dx + \int_{\Omega_N} h N_i ds  \quad \forall j .    
\]
A simple rewrite: 
\[
\sum_j u_j  \int_\Omega k \nabla  N_j  \cdot \nabla N_i dx = \int_\Omega f N_i dx + \int_{\Omega_N} h N_i ds \quad \forall j.    
\]
Hence, with  
\begin{eqnarray*}
A_{ij} = \int_\Omega k \nabla  N_j  \cdot \nabla N_i dx, \\  
b_i = \int_\Omega f N_i dx + \int_{\Omega_N} h N_i ds 
\end{eqnarray*}
we arrive at the following linear system 
\[
A u = b 
\]

The following code solves the Poisson problem on the unit square
consisting of $32\times 32$ rectangles, where each rectangle is divided
in two and $f=1$, $g=0$ and $h=x$. Dirichlet conditions are set 
for $y=0$ and Neumann for the rest of $\partial \Omega$.  

\begin{python}
from dolfin import *

# Create mesh and define function space
mesh = UnitSquareMesh(32, 32)
V = FunctionSpace(mesh, "Lagrange", 1)

# Define Dirichlet boundary (x = 0 or x = 1)
def boundary(x): return x[0] < DOLFIN_EPS 

# Define boundary condition
u0 = Constant(0.0)
bc = DirichletBC(V, g, boundary)

# Define variational problem
u = TrialFunction(V)
v = TestFunction(V)
f = Constant(1)
g = Expression("x[0]")
a = inner(grad(u), grad(v))*dx
L = f*v*dx + h*v*ds

# Compute solution
u = Function(V)
solve(a == L, u, bc)

# Save solution in VTK format
file = File("poisson.pvd")
file << u
\end{python}



A fundamental property in both the theory of partial differential equations and numerical analysis is the
concept of well-posedness. In Hadamard´s definition a problem is well-posed if a solution exists, is unique
and depends continuously on the input. Hence, if we have two inputs to our problem, $b_1$ and $b_2$ 
then the difference between the solutions $u_1$ and $u_2$ should be bounded by the differences
between $b_1$ and $b_2$. In terms of linear algebra, we directly obtain well-posedness if
$A$ is a non-singular matrix. That we directly have 
\[
\|A(u_1 - u_2) \| = \|b_1 - b_2\|     
\]
and 
\[
\|(u_1 - u_2) \| = \|A^{-1}b_1 - b_2\|     
\]
which we directly can explore to obtain bounds. Neither of these observations are however
very useful. However, assuming that we have a concept of 
$A^{1/2}$ and $A^{-1/2}$, which will be more precisely defined later, 
we will see that the notion 
\[
\|A^{1/2}(u_1 - u_2) \| = \|A^{-1/2}b_1 - b_2\|     
\]
is extraordinary useful and gives precise estimates in a wide range of situations.  
We remark that $\Delta=\nabla\cdot\nabla$ and as such $\nabla$ can be interpreted
as some kind of square root of $\Delta$. However, these things will need to be made
more precise.  






